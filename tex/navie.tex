\documentclass[a4paper,11pt]{article}
\usepackage[english,russian]{babel}
\usepackage[utf8]{inputenc}
\usepackage{a4wide}
\usepackage{amsmath}
\usepackage{float}
\usepackage{graphicx}

\begin{document}

\section{Начально-краевая задача}
\label{dif}

Изучение свойств численного алгоритма и создание его программной реализации
начинается с наиболее простых задач с постепенным их усложнением.
Приведем систему уравнений, описывающую нестационарное
одномерное движение вязкого баротропного газа

\begin{equation}
\begin{array}{l}
\dfrac{\partial\rho}{\partial t} + \dfrac{\partial\rho u}{\partial x} = 0, \\[2.0ex]
\rho\dfrac{\partial u}{\partial t} + \rho u\dfrac{\partial u}{\partial x}  +
\dfrac{\partial p}{\partial x} =\mu\dfrac{\partial^2 u}{\partial x^2}  + \rho f, \\[1.5ex]
p=p(\rho).
\end{array}
\label{onedim.1}
\end{equation}

Выше через $\mu$ обозначен коэффициент
вязкости газа, который будем считать
известной неотрицательной константой.

Неизвестные функции: плотность $\rho$  и  скорость $u$ являются
функциями переменных Эйлера
$$
(t,x) \in Q=[0,T]\times [0,X].
$$%

В уравнения входят еще две известные функции:
давление газа $p$,  зависящее от плотности, и вектор внешних сил $f$,
являющийся функцией переменных Эйлера.

В начальный момент времени задаются функции, значениями которых являются
плотность и скорость газа в точках отрезка $[0,\,X]$:
\begin{equation}
(\rho,u)|_{t=0} =(\rho_0,u_0),\;  x\in [0,X].
\label{onedim.2}
\end{equation}

Простейшими граничными условиями являются условия непротекания
\begin{equation}
u(t,0)=u(t,X)=0,\;  t\in [0,T].
\label{onedim.3}
\end{equation}
В этом случае граничные условия на плотность газа не ставятся.

Дифференциальные уравнения системы этой системы являются следствиями
интегральных законов сохранения массы и импульса в случае достаточной
гладкости функций плотности и скорости газа.
Важными требованиями к вычислительным алгоритмам являются выполнение аналогов
этих законов для сеточных функций. Численные методы, для которых выполняются
один или несколько законов сохранения (но не все), называют консервативными.
Если выполняются все законы сохранения, то метод называют полностью
консервативным.

\newpage
\section{Основные обозначения}
\label{obozn}
В заданиях практикума рассматривается
пространственные области в виде отрезка
$\bar\Omega =[0;\,X]$, где вводится равномерная сетка с шагом $h$:
$\bar\omega_{h}=\{mh\mid m=0,\dots,M\}$, где $Mh=X$. На временном
интервале $[0;\,T]$ также используется равномерная сетка:
$\omega_{\tau}=\{n\tau\mid n=0,\dots,N\},$ где $N\tau =T$. В результате в
области $Q$ вводится сетка $\bar Q_{\tau h}=\omega_{\tau}\times\bar\omega_{\bar h}$.
Узлы сетки
$\bar\omega_{h}$, попадающие на границу области $\omega$, обозначим $\gamma_{h}$
(граничные узлы), а попадающие в область $\omega$ через $\omega_{h}$ (внутренние
узлы). Узелы $x=0$ и $x=X$ будем обозначать $\gamma^{-}_{h}$ и $\gamma^{+}_{h}$
соответственно.

Кроме сетки $\omega_{h}$ в ряде схем используются сдвинутые сетки с
полуцелыми узлами. Через $\omega^{1/2}_{h}$ будем обозначать сетку
$\omega^{1/2}_{h}=\{mh+h/2\mid m=0,\dots,M-1\}$, а через
$Q^{1/2}_{\tau h}=\omega_{\tau}\times\omega^{1/2}_{h}$.

Значение функции $g$, определенной на сетке $Q_{\tau h}$
(или на сетке $Q^{1/2}_{\tau h}$), в узле $(n, m)$
будем обозначать через $g^n_{m}$. Если индексы будут опущены, то это
означает, что они равны $n$ и $m$. Для сокращения записи значений функции
$g$ в узлах, соседних с узлом $(n,m)$, используются следующие обозначения:
$$
g^{n+1}_{m}=\hat g, \qquad g^{n}_{m\pm 1}= g^{\pm 1}.
$$%
Введем обозначения для среднего значения величин сеточной функции в двух
соседних узлах
$$
g_{s}=\dfrac{g^n_{m+1} +g^n_{m}}{2},\quad
g_{\bar s_k}=\dfrac{g^n_{m}+g^n_{m-1}}{2}.
$$

Для разностных операторов применяются следующие обозначения:
$$
g_t=\dfrac{ g^{n+1}_{m} - g^n_{m}}{\tau},\quad
g_{x}=\dfrac{g^n_{m+1} -g^n_{m}}{h},\quad
g_{{\mathop{x}\limits^\circ}}=\dfrac{g^n_{m+1}-g^n_{m-1}}{2h}, \quad
g_{\bar x}=\dfrac{g^n_{m}-g^n_{m-1}}{h}.
$$%

В схеме А.Г.Соколова в конвективных слагаемых узел шаблона, в котором нужно
брать значение сеточной функции, зависит от знака компоненты вектора скорости.
Для этих выражений используется обозначение
\begin{equation}
\sigma\{H,V\}=H\dfrac{|V|-V}{2|V|}+H^{(-1)}\dfrac{|V|+V}{2|V|}=\left\{
\begin{array}{l}
H,\qquad \hbox{если}\;V<0,\\
H^{(-1)},\; \hbox{если}\;V\ge 0,
\end{array}
\right.
\end{equation}

\newpage
\section{Разностная схема А.Г.Соколова ПЛОТНОСТЬ-СКОРОСТЬ}
Функции $H$, заданная в узлах сетки $\omega^{1/2}_h$, и
$V$, заданная в узлах сетки $\omega_h$, ищутся по схеме:
\begin{equation}
H_t+(\sigma\{\hat H,V\} V)_x=0, \qquad 0\le m <M,\; n\ge 0,\\
\end{equation}
\begin{equation}
\begin{array}{l}
\hat H_{\bar s}V_t+
\hat H_{\bar s}\delta\{\hat V,V\}
+\dfrac{\gamma}{\gamma -1}\hat H_{\bar s}((\hat H)^{\gamma-1})_{\bar x}
=\mu\hat V_{x\bar x}+\hat H_{\bar s}f,
\qquad \hbox{при}\; \hat H_{\bar s}\ne 0,\\
\hat V=0, \qquad \hbox{при}\; \hat H_{\bar s}=0,\\
0< m <M,\; n\ge 0,\\
\hat V_0=\hat V_M=0.
\end{array}
\end{equation}

Разностные уравнения в индексах имеют вид:
\begin{equation}
\begin{array}{c}
\dfrac{H^{n+1}_m-H^n_m}{\tau}+\\ +\dfrac{
(V^n_{m+1}-|V^n_{m+1}|)H_{m+1}^{n+1}+
(V^n_{m+1}+|V^n_{m+1}|-V^n_{m}+|V^n_{m}|)H_{m}^{n+1}-
(V^n_{m}+|V^n_{m}|)H_{m-1}^{n+1}
}{2h}=0, \\
0\le m <M,\; n\ge 0.
\end{array}
\end{equation}
\begin{equation}
\begin{array}{c}
\dfrac{(H^{n+1}_{m-1}+H^{n+1}_m)V^{n+1}_m-(H^n_{m-1}+H^n_m)V^n_m}{2\tau}-\\
-\dfrac{
((|V^n_{m-1}|+V^n_{m-1})H^{n+1}_{m-2}+
(|V^n_{m}|+V^n_{m})H_{m-1}^{n+1})V^{n+1}_{m-1}}{4h}+\\
+\dfrac{((|V^n_{m-1}|-V^n_{m-1}+|V^n_m|+V^n_m)H_{m-1}^{n+1}+
(|V^n_{m+1}|+V^n_{m+1}+|V^n_m|-V^n_m)H_m^{n+1})V^{n+1}_m}{4h}-\\
-\dfrac{((|V^n_m|-V^n_m)H_m^{n+1}+(|V^n_{m+1}|-V^n_{m+1})H_{m+1}^{n+1})V^{n+1}_{m+1}
}{4h}+ \\
+\dfrac{\gamma}{\gamma -1}\dfrac{H^{n+1}_{m}+H^{n+1}_{m-1}}{2}
\dfrac{(H^{n+1}_{m})^{\gamma-1}-(H^{n+1}_{m-1})^{\gamma-1}}{h}= \\
=\mu\dfrac{V^{n+1}_{m-1}-2V^{n+1}_{m}+V^{n+1}_{m+1}}{h^2}
+\dfrac{H^{n+1}_{m-1}+H^{n+1}_m}{2}f_m^{n+1},\\
\qquad \hbox{при}\; H^{n+1}_{m-1}+H^{n+1}_m\ne 0,\\
V^{n+1}_m=0, \qquad \hbox{при}\; H^{n+1}_{m-1}+H^{n+1}_m=0,\\
0< m <M,\; n\ge 0,\\
V^{n+1}_0=V^{n+1}_M=0.
\end{array}
\end{equation}

\newpage
\section{Заполнение матрицы первой системы}
За заполнение матрицы первой системы отвечает следующий код:
\begin{verbatim}
void fill_first (std::vector<double> &A, std::vector<double> &B,
                 discrete_function &H, discrete_function &V,
                 int n, double h, double tau, discrete_function &/*f*/,
                 discrete_function &f_0)
{
  std::vector<double> &H_cut = H.cut (n);
  std::vector<double> &V_cut = V.cut (n);
  unsigned int M = static_cast<unsigned int> (H_cut.size ());

}
\end{verbatim}

\newpage
\section{Таблицы невязок в непрерывном случае:}
\subsection{MIU = 0.100}
\begin{table}[H]
\caption {Функция: H Тип невязки: C   }
\begin{center}
\begin{tabular}{l|l|l|l|l}
\hline
M/N  & 21 & 42 & 84 & 168 \\ \hline
  21 & 1e+00& 7e-01& 4e-01& 1e+01\\ \hline
  42 & 1e+00& 7e-01& 4e-01& 2e-01\\ \hline
  84 & 1e+00& 6e-01& 4e-01& 2e-01\\ \hline
 168 & 1e+00& 6e-01& 4e-01& 2e-01\\ \hline
\end{tabular}
\end{center}
\end{table}
\begin{table}[H]
\caption {Функция: V1 Тип невязки: C   }
\begin{center}
\begin{tabular}{l|l|l|l|l}
\hline
M/N  & 21 & 42 & 84 & 168 \\ \hline
  21 & 6e-01& 3e-01& 2e-01& 1e+00\\ \hline
  42 & 7e-01& 4e-01& 2e-01& 8e-02\\ \hline
  84 & 7e-01& 4e-01& 2e-01& 1e-01\\ \hline
 168 & 7e-01& 4e-01& 2e-01& 1e-01\\ \hline
\end{tabular}
\end{center}
\end{table}
\begin{table}[H]
\caption {Функция: V2 Тип невязки: C   }
\begin{center}
\begin{tabular}{l|l|l|l|l}
\hline
M/N  & 21 & 42 & 84 & 168 \\ \hline
  21 & 2e-01& 1e-01& 1e-01& 1e+00\\ \hline
  42 & 2e-01& 1e-01& 8e-02& 5e-02\\ \hline
  84 & 2e-01& 1e-01& 8e-02& 4e-02\\ \hline
 168 & 2e-01& 1e-01& 8e-02& 4e-02\\ \hline
\end{tabular}
\end{center}
\end{table}
\begin{table}[H]
\caption {Функция: H Тип невязки: L2  }
\begin{center}
\begin{tabular}{l|l|l|l|l}
\hline
M/N  & 21 & 42 & 84 & 168 \\ \hline
  21 & 3e+00& 2e+00& 1e+00& 9e+00\\ \hline
  42 & 3e+00& 1e+00& 9e-01& 7e-01\\ \hline
  84 & 3e+00& 1e+00& 7e-01& 5e-01\\ \hline
 168 & 2e+00& 1e+00& 7e-01& 4e-01\\ \hline
\end{tabular}
\end{center}
\end{table}
\begin{table}[H]
\caption {Функция: V1 Тип невязки: L2  }
\begin{center}
\begin{tabular}{l|l|l|l|l}
\hline
M/N  & 21 & 42 & 84 & 168 \\ \hline
  21 & 2e+00& 8e-01& 3e-01& 9e-01\\ \hline
  42 & 2e+00& 1e+00& 5e-01& 2e-01\\ \hline
  84 & 2e+00& 1e+00& 5e-01& 2e-01\\ \hline
 168 & 2e+00& 1e+00& 6e-01& 3e-01\\ \hline
\end{tabular}
\end{center}
\end{table}
\begin{table}[H]
\caption {Функция: V2 Тип невязки: L2  }
\begin{center}
\begin{tabular}{l|l|l|l|l}
\hline
M/N  & 21 & 42 & 84 & 168 \\ \hline
  21 & 7e-01& 4e-01& 3e-01& 1e+00\\ \hline
  42 & 7e-01& 4e-01& 2e-01& 1e-01\\ \hline
  84 & 7e-01& 4e-01& 2e-01& 1e-01\\ \hline
 168 & 6e-01& 4e-01& 2e-01& 1e-01\\ \hline
\end{tabular}
\end{center}
\end{table}
\subsection{MIU = 0.010}
\begin{table}[H]
\caption {Функция: H Тип невязки: C   }
\begin{center}
\begin{tabular}{l|l|l|l|l}
\hline
M/N  & 21 & 42 & 84 & 168 \\ \hline
  21 & 1e+00& 7e-01& 4e-01& 7e+01\\ \hline
  42 & 1e+00& 7e-01& 4e-01& 2e-01\\ \hline
  84 & 1e+00& 7e-01& 4e-01& 2e-01\\ \hline
 168 & 1e+00& 7e-01& 4e-01& 2e-01\\ \hline
\end{tabular}
\end{center}
\end{table}
\begin{table}[H]
\caption {Функция: V1 Тип невязки: C   }
\begin{center}
\begin{tabular}{l|l|l|l|l}
\hline
M/N  & 21 & 42 & 84 & 168 \\ \hline
  21 & 7e-01& 3e-01& 2e-01& 4e+00\\ \hline
  42 & 7e-01& 4e-01& 2e-01& 9e-02\\ \hline
  84 & 7e-01& 4e-01& 2e-01& 1e-01\\ \hline
 168 & 7e-01& 4e-01& 2e-01& 1e-01\\ \hline
\end{tabular}
\end{center}
\end{table}
\begin{table}[H]
\caption {Функция: V2 Тип невязки: C   }
\begin{center}
\begin{tabular}{l|l|l|l|l}
\hline
M/N  & 21 & 42 & 84 & 168 \\ \hline
  21 & 3e-01& 2e-01& 1e-01& 6e+00\\ \hline
  42 & 3e-01& 2e-01& 9e-02& 6e-02\\ \hline
  84 & 3e-01& 2e-01& 9e-02& 5e-02\\ \hline
 168 & 3e-01& 2e-01& 9e-02& 5e-02\\ \hline
\end{tabular}
\end{center}
\end{table}
\begin{table}[H]
\caption {Функция: H Тип невязки: L2  }
\begin{center}
\begin{tabular}{l|l|l|l|l}
\hline
M/N  & 21 & 42 & 84 & 168 \\ \hline
  21 & 3e+00& 2e+00& 1e+00& 3e+01\\ \hline
  42 & 3e+00& 1e+00& 9e-01& 7e-01\\ \hline
  84 & 3e+00& 1e+00& 8e-01& 5e-01\\ \hline
 168 & 3e+00& 1e+00& 7e-01& 4e-01\\ \hline
\end{tabular}
\end{center}
\end{table}
\begin{table}[H]
\caption {Функция: V1 Тип невязки: L2  }
\begin{center}
\begin{tabular}{l|l|l|l|l}
\hline
M/N  & 21 & 42 & 84 & 168 \\ \hline
  21 & 2e+00& 9e-01& 4e-01& 3e+00\\ \hline
  42 & 2e+00& 1e+00& 5e-01& 2e-01\\ \hline
  84 & 2e+00& 1e+00& 5e-01& 2e-01\\ \hline
 168 & 2e+00& 1e+00& 6e-01& 3e-01\\ \hline
\end{tabular}
\end{center}
\end{table}
\begin{table}[H]
\caption {Функция: V2 Тип невязки: L2  }
\begin{center}
\begin{tabular}{l|l|l|l|l}
\hline
M/N  & 21 & 42 & 84 & 168 \\ \hline
  21 & 7e-01& 4e-01& 3e-01& 8e+00\\ \hline
  42 & 7e-01& 4e-01& 2e-01& 2e-01\\ \hline
  84 & 7e-01& 4e-01& 2e-01& 1e-01\\ \hline
 168 & 7e-01& 4e-01& 2e-01& 1e-01\\ \hline
\end{tabular}
\end{center}
\end{table}
\subsection{MIU = 0.001}
\begin{table}[H]
\caption {Функция: H Тип невязки: C   }
\begin{center}
\begin{tabular}{l|l|l|l|l}
\hline
M/N  & 21 & 42 & 84 & 168 \\ \hline
  21 & 1e+00& 7e-01& 4e-01& 1e+02\\ \hline
  42 & 1e+00& 7e-01& 4e-01& 2e-01\\ \hline
  84 & 1e+00& 7e-01& 4e-01& 2e-01\\ \hline
 168 & 1e+00& 7e-01& 4e-01& 2e-01\\ \hline
\end{tabular}
\end{center}
\end{table}
\begin{table}[H]
\caption {Функция: V1 Тип невязки: C   }
\begin{center}
\begin{tabular}{l|l|l|l|l}
\hline
M/N  & 21 & 42 & 84 & 168 \\ \hline
  21 & 7e-01& 3e-01& 2e-01& 4e+00\\ \hline
  42 & 7e-01& 4e-01& 2e-01& 9e-02\\ \hline
  84 & 7e-01& 4e-01& 2e-01& 1e-01\\ \hline
 168 & 7e-01& 4e-01& 2e-01& 1e-01\\ \hline
\end{tabular}
\end{center}
\end{table}
\begin{table}[H]
\caption {Функция: V2 Тип невязки: C   }
\begin{center}
\begin{tabular}{l|l|l|l|l}
\hline
M/N  & 21 & 42 & 84 & 168 \\ \hline
  21 & 3e-01& 2e-01& 1e-01& 5e+00\\ \hline
  42 & 3e-01& 2e-01& 9e-02& 6e-02\\ \hline
  84 & 3e-01& 2e-01& 9e-02& 5e-02\\ \hline
 168 & 3e-01& 2e-01& 9e-02& 5e-02\\ \hline
\end{tabular}
\end{center}
\end{table}
\begin{table}[H]
\caption {Функция: H Тип невязки: L2  }
\begin{center}
\begin{tabular}{l|l|l|l|l}
\hline
M/N  & 21 & 42 & 84 & 168 \\ \hline
  21 & 3e+00& 2e+00& 1e+00& 3e+01\\ \hline
  42 & 3e+00& 1e+00& 9e-01& 7e-01\\ \hline
  84 & 3e+00& 1e+00& 8e-01& 5e-01\\ \hline
 168 & 3e+00& 1e+00& 7e-01& 4e-01\\ \hline
\end{tabular}
\end{center}
\end{table}
\begin{table}[H]
\caption {Функция: V1 Тип невязки: L2  }
\begin{center}
\begin{tabular}{l|l|l|l|l}
\hline
M/N  & 21 & 42 & 84 & 168 \\ \hline
  21 & 2e+00& 9e-01& 4e-01& 4e+00\\ \hline
  42 & 2e+00& 1e+00& 5e-01& 2e-01\\ \hline
  84 & 2e+00& 1e+00& 6e-01& 3e-01\\ \hline
 168 & 2e+00& 1e+00& 6e-01& 3e-01\\ \hline
\end{tabular}
\end{center}
\end{table}
\begin{table}[H]
\caption {Функция: V2 Тип невязки: L2  }
\begin{center}
\begin{tabular}{l|l|l|l|l}
\hline
M/N  & 21 & 42 & 84 & 168 \\ \hline
  21 & 7e-01& 5e-01& 3e-01& 9e+00\\ \hline
  42 & 7e-01& 4e-01& 2e-01& 2e-01\\ \hline
  84 & 7e-01& 4e-01& 2e-01& 1e-01\\ \hline
 168 & 7e-01& 4e-01& 2e-01& 1e-01\\ \hline
\end{tabular}
\end{center}
\end{table}

\subsection{Стабилизация течения в разрывном случае:}

\subsection{Примеры графиков, непрерывный случай:}
\includegraphics[width=1.0\textwidth]{cont.png}

\subsection{Примеры графиков, разрывный случай}
\includegraphics[width=1.0\textwidth]{cont.png}

\section{Выводы}

\end{document}

